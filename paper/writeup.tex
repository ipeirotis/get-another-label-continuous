\documentclass{article}

\begin{document}

\title{Quality control when measuring a continuous variable with noisy workers}

\maketitle

\section{Introduction}

In crowdsourcing settings, we have access to a large number of people that provide answers to a variety of tasks. A key problem in such settings is that we do not know the quality of the answers provided by the participants. The key solutions proposed over the last few years all revolve around two key ideas: redundancy and gold labels.  However, most of the quality control techniques are designed to operate with discrete answers. In this paper, we introduce a technique that, in the presence of noisy workers, and answers that are a continuous variable
\begin{itemize}
\item Estimates the most likely answer values
\item Estimates the quality of the workers that return the answers
\end{itemize}

\section{Model Assumptions}

The basic idea is the following: The true answers $X = \{ x_1, \ldots, x_n \}$ are drawn from a latent distribution $F(\theta)$; we cannot observe directly the values of $X$. Instead, we have a set of noisy labelers $w^1, \ldots, w^k$; each worker $w_j$ observes one or more of the examples in $X$ and returns a noisy estimate of the observed value. For notational purposes, we use 
$y_j(x_i)$ to mark the value that worker $w_j$ returns after observing the value $x_i$. We use $G_j(\theta_j)$ to denote the distribution of returned values for worker $w_j$. The distribution $G_j(\theta_j)$  is independent of $F(\theta)$ if the worker $w_j$ returns random results. If $F(\theta)$ is identical with $G_j(\theta_j)$, the worker is perfect (we can also have cases where $G_j(\theta_j)$ is not identical with $F(\theta)$ and the worker is still perfect). We assume that the distributions $G_j(\theta_j)$ are conditional independent of each other, given $F(\theta)$ (i.e., the workers only observe the ``latent'' values $x_i$ and the value $y_j(x_i)$ is only influenced by $x_i$ and the quality of the worker $w_j$, and not by the values assigned by other workers).

\textbf{The bi-variate Gaussian case:} We restrict our current approach to the case where $F(\theta)$ and $G_j(\theta_j)$ form a bi-variate Gaussian distribution; the quality of the worker $w_j$ is the correlation $\rho_j$ between $F(\theta)$ and $G_j(\theta_j)$. Since $F$ and $G_j$ are two Gaussians, their parameters are $\theta={\mu, \sigma}$ and $\theta_j={\mu_j, \sigma_j}$.

[TODO: Add a picture of the correlated distributions, showing the effect of $\rho$ in the bi-variate distribution]


\section{Estimation Algorithm I: Point-to-point}

Our goal is to know the quality $\rho_j$ (the covariance) of each worker, and the most likely correct value  $x_i$ for each data point. 

We start by estimating the characteristics of the distribution $G_j$ of the answers from each worker $w_j$. Given the points $Y_j$ returned by worker $w_j$, we estimate the mean $\mu_j$ and variance $\sigma_j$ of the $G_j$ distribution using the usual maximum likelihood estimates:

\begin{eqnarray}
\mu_j = \frac{1}{|Y_j|}\cdot\sum_{y_j(x_i) \in Y_j} y_j(x_i) \label{equ:muj} \\
\sigma_j= \frac{1}{|Y_j|}\cdot\sum_{y_j(x_i) \in Y_j} (y_j(x_i) - \mu_j)^2 \label{equ:sigmaj}
\end{eqnarray}

Now, to compute the reliability $\rho_j$ of each worker, we need to know the matching $x_i$ values for each $y_j(x_i)$ label assigned by the worker. We have:

\begin{eqnarray}
P ( X | Y_1 \ldots Y_k) &=&  \frac{ P (  Y_1 \ldots Y_k | X ) \cdot P(X)} {P (  Y_1 \ldots Y_k)} \nonumber \\
&=&  \frac{ P(X) \cdot \prod_{j=1}^k P (  Y_j | X )  } {\prod_{j=1}^k P (  Y_j )} \nonumber \\
&=&  \frac{ P(X) \cdot \prod_{j=1}^k P (  X | Y_j ) P(Y_j) / P(X) } {\prod_{j=1}^k P (  Y_j )} \nonumber \\
&\propto&  \prod_{j=1}^k P (  X | Y_j ) \nonumber
\end{eqnarray}

Now, we further break $P(X|Y_j)$ by splitting $X$ into its individual points, and now each point $x_i$ gets conditioned on the overall vector $Y_j$ and the estimate $y_j(x_i)$ for $x_i$ given by worker $w_j$.

\begin{equation}
\label{equ:pxgiveny}
P ( X | Y_1 \ldots Y_k) \propto  \prod_{j=1}^k \prod_{i=1}^n P (  x_i | y_j(x_i), Y_j ) 
\end{equation}

To estimate $P (  x_i | y_j(x_i), Y_j ) $ we rely on the fact that $X$ and $Y_j$ (and therefore $x_i$ and $y_j(x_i)$) are drawn from a bivariate normal distribution. So, the distribution of the conditional $P (  x_i | y_j(x_i), Y_j  )$ is a Gaussian distribution with mean and variance given below:

\begin{eqnarray}
P (  x_i | y_j(x_i), Y_j  ) &=& \mathcal{N}(x_i; \widehat{\mu_j} ,\widehat{\sigma_j} ) \nonumber \\
\mathcal{N}(x; \mu, \sigma) & = & \frac{1}{\sigma\sqrt{2\pi}} \cdot \exp{ -\frac{(x-\mu)^2}{2\sigma^2} } \nonumber \\
\widehat{\mu_j} &= &\mu + \rho_j \frac{\sigma}{\sigma_j} (y_j(x_i) - \mu_j) \nonumber \\
\widehat{\sigma_j} &=&\sqrt{1-\rho_j^2} \cdot \sigma \nonumber 
\end{eqnarray}

We now identify the most likely values in $X$, given the label assignments by the workers. For that, we compute the log-likelihood of the probability:

\begin{eqnarray}
L = -\ln\left(P ( X | Y_1 \ldots Y_k)\right) 
&\propto&  \sum_{j=1}^k \sum_{i=1}^n \ln(P (  x_i | y_j(x_i), Y_j )) \nonumber \\
&\propto&  \sum_{j=1}^k \sum_{i=1}^n \left( \frac{\ln(2\pi)}{2}+ \ln(\widehat{\sigma_j}) + \frac{(x_i-\widehat{\mu_j})^2}{2\widehat{\sigma_j}^2} \right) \nonumber 
\end{eqnarray}

Given that the values $x_i$ are generated in an i.i.d.\ fashion, we can find the $X$ that maximizes the likelihood by taking the partial derivative for each $x_i$ and setting it to zero. This will return the ``most likely'' $x_i$ values, given the values returned by the noisy workers.


%\begin{eqnarray}
%\frac{\partial L}{\partial x_i} = \sum_{j=1}^k \frac{\partial}{\partial x_i} \left( \frac{(x_i-\mu - \rho_j \frac{\sigma}{\sigma_j} (y_j(x_i) - \mu_j))^2}{2\widehat{\sigma_j}^2} \right) & = & 0 \nonumber \\
%\sum_{j=1}^k  \left( \frac{(x_i-\mu - \rho_j \frac{\sigma}{\sigma_j} (y_j(x_i) - \mu_j)) \cdot (1 - \rho_j \frac{\sigma}{\sigma_j} (\frac{\partial y_j(x_i)}{\partial x_i})) }{\widehat{\sigma_j}^2} \right) & = & 0 \nonumber \\
%\sum_{j=1}^k  \left( \frac{(x_i-\mu - \rho_j \frac{\sigma}{\sigma_j} (y_j(x_i) - \mu_j)) }{\widehat{\sigma_j}^2} \cdot (1 - \rho_j^2 ) \right) & = & 0 \nonumber \\
%\sum_{j=1}^k \left( \frac{x_i-\mu - \rho_j \frac{\sigma}{\sigma_j} (y_j(x_i) - \mu_j)}{\widehat{\sigma_j}^2} \cdot (1 - \rho_j^2 ) \right) & = & 0 \nonumber \\
%\sum_{j=1}^k \left( \frac{x_i-\mu - \rho_j \frac{\sigma}{\sigma_j} (y_j(x_i) - \mu_j)}{(1-\rho_j^2) \cdot \sigma^2}  \cdot (1 - \rho_j^2 ) \right) & = & 0 \nonumber \\
%\sum_{j=1}^k \left( \frac{x_i-\mu}{\sigma} - \rho_j \frac{y_j(x_i) - \mu_j}{\sigma_j} \right) & =&  0 \label{equ:diff}
%\end{eqnarray}

%\noindent where we set $\frac{\partial y_j(x_i)}{\partial x_i} = \rho_j \frac{\sigma_j}{\sigma}$. (\textbf{TODO: Double check this part, it is important}).


\begin{eqnarray}
\frac{\partial L}{\partial x_i} = \sum_{j=1}^k \frac{\partial}{\partial x_i} \left( \frac{(x_i-\mu - \rho_j \frac{\sigma}{\sigma_j} (y_j(x_i) - \mu_j))^2}{2\widehat{\sigma_j}^2} \right) & = & 0 \nonumber \\
\sum_{j=1}^k  \left( \frac{x_i-\mu - \rho_j \frac{\sigma}{\sigma_j} (y_j(x_i) - \mu_j) }{\widehat{\sigma_j}^2} \right) & = & 0 \nonumber \\
\sum_{j=1}^k \left( \frac{x_i-\mu - \rho_j \frac{\sigma}{\sigma_j} (y_j(x_i) - \mu_j)}{\widehat{\sigma_j}^2} \right) & = & 0 \nonumber \\
\sum_{j=1}^k \left( \frac{x_i-\mu - \rho_j \frac{\sigma}{\sigma_j} (y_j(x_i) - \mu_j)}{(1-\rho_j^2) \cdot \sigma^2} \right) & = & 0 \nonumber \\
\sum_{j=1}^k \left( \frac{ \frac{x_i-\mu}{\sigma} - \rho_j \frac{y_j(x_i) - \mu_j}{\sigma_j}}{(1-\rho_j^2)} \right) & = & 0 \label{equ:diff} \\
\end{eqnarray}



To simplify the notation, we now set:

\begin{eqnarray}
z_i	& = & \frac{x_i - \mu}{\sigma} \label{equ:zi} \\
z_i^j	& = & \frac{y_j(x_i) - \mu_j}{\sigma_j} \label{equ:zij} \\
\beta_j & = & \frac{1}{1-\rho_j^2} \label{equ:betaj} \\
\rho_j  & = & \sqrt{1-\frac{1}{\beta_j}} \label{equ:rhoj}
\end{eqnarray}

\noindent Using the equations above, and substituting in Equation~\ref{equ:diff}, we have:

\begin{displaymath}
\sum_{j=1}^k \beta_j \cdot \left( z_i - \rho_j \cdot z_i^j \right) = 0
\end{displaymath}


%\begin{displaymath}
%\sum_{j=1}^k \left( z_i - \rho_j \cdot z_i^j \right) = 0
%\end{displaymath}

\noindent Solving for $z_i$, we get:

\begin{equation}
\label{equ:z-aggregate}
z_i = \frac{ \sum_{j=1}^k \beta_j \cdot \rho_j \cdot z_i^j } {\sum_{j=1}^k \beta_j } 
\end{equation}

\noindent As it becomes clear from thiq equation, the best point estimate that we have for $z_i$ is a weighted average of the $\rho_j \cdot z_i^j$ values, and the weight of each contribution is equal to $\rho_j$ (see Equation~\ref{equ:betaj}). It is worth noting that the $\rho_j \cdot z_i^j$ is the maximum likelihood estimate for the value of $z_i$ when we have a single measurement from a worker with correlation $\rho_j$ who returns a measurement equal to $z_i^j$.

\begin{itemize}
\item TODO: Baseline error: The baseline absolute error can be computed as the expected abso

\item TODO: The expected error of $z_i$, given a distribution of $\rho_j$ values with pdf $f(\rho)$, is $\int f(\rho) ....$.

\item TODO: Estimate the variance of $z_i$ for a given set of $\rho_j$ values.

\item TODO: Estimate the variance of $z_i$ for a probability distribution of $\rho_j$ values.

\end{itemize}


Knowing the (normalized) values of the latent variables $z_i$ and the assigned labels $z_i^j$ from the workers,\footnote{Notice that the $z_i$ and $z^j_i$ variables have a zero mean, so in Equation~\ref{equ:worker-quality}, there  is no need to subtract the mean and the equation from the $z$ values, and the equation is simplified. Of course, someone can always take the extra step of computing the empirical mean and standard deviation for the $z_i$ and $z^j_i$ values.} we can now estimate the quality $\rho_j$ of each worker as:

\begin{equation}
\label{equ:worker-quality}
\rho_j = \frac{ \sum_{i=1}^n z_i \cdot z^j_i }{ \sqrt{\sum_{i=1}^n (z_i)^2 \cdot  \sum_{i=1}^n (z^j_i)^2 }}
\end{equation}

At this point, we have estimated the quality of the different labelers, and we have the \emph{standardized} values $z_i$ from the original distribution. Note that, at this point, we can use \emph{any} value $\mu$ and $\sigma$ to generate the values $x_i$ from the latent distribution $F$, and the solution will be mathematically fine. (In other words, the model is not purely ``identifiable.'') 

As a workaround, we assume that the crowd \emph{after quality-adjustment}, does not have a systematic bias. So, we can set the mean value $\mu$ for $F$, as the combination of the mean values from the $G_j$ distributions, adjusted for the reliability of each worker, in the same way that we estimate $x_i$'s by combining the values provided by the different workers.

\begin{equation}
\label{equ:mu}
\mu = \frac{ \sum_{j=1}^k \beta_j \cdot \sqrt{1-\frac{1}{\beta_j}} \cdot \mu_j } {\sum_{j=1}^k \beta_j} 
\end{equation}

\noindent Similarly for standard deviation $\sigma$:

\begin{equation}
\label{equ:sigma}
\sigma = \frac{ \sum_{j=1}^k \beta_j \cdot \sqrt{1-\frac{1}{\beta_j}} \cdot \sigma_j } {\sum_{j=1}^k \beta_j} 
\end{equation}

\noindent Finally, we transform now each $z_i$ value into the most likely estimate $x_i$:

\begin{equation}
\label{equ:xi}
x_i = \sigma \cdot z_i + \mu
\end{equation}

\emph{Note: These equations give equal weight to labelers with same quality, independent of the number of labeled examples. This can be both good and bad. Good because no single labeler can influence the outcome, bad because we trust equally the $\mu_j$ and $\sigma_j$ estimates from workers with large and with small number of submissions, while the estimates derived from large number of submissions are by definition more robust.}


\subsection{Algorithm}

\begin{enumerate}

\item For each worker $w_j$, with submissions $Y_j = \{ y_j(x_i) \}$, compute the mean $\mu_j$ using Equation~\ref{equ:muj}, and the standard deviation $\sigma_j$ using Equation~\ref{equ:sigmaj}.

\item For each worker $w_j$, transform the submissions $y_j(x_i)$ into standardized (zero-mean, standard deviation one) $z$-scores, using Equation~\ref{equ:zij}.

\item Assume equal quality $\rho_j$ for all workers\footnote{Any value $0<\rho_j<1$ should work as an initial condition, but \emph{not} the degenerate values 0 or 1.} and then compute the coefficients $\beta_j$ using Equation~\ref{equ:betaj}.

\item \label{step:e} Use Equation~\ref{equ:z-aggregate}, and compute the most likely standardized score $z_i$ for each labeled example.

\item \label{step:m} Using the computed $z_i$ values, re-compute the quality $\rho_j$ of each worker using Equation~\ref{equ:worker-quality} and recompute the coefficients $\beta_j$.

\item Repeat Steps~\ref{step:e} and~\ref{step:m} until convergence.

\item Use Equations~\ref{equ:mu}, \ref{equ:sigma}, and~\ref{equ:xi} to transform the $z_i$ values into the most likely estimates $x_i$ of the latent variables.

\end{enumerate}



\end{document}


